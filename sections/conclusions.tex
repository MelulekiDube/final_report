\section{Conclusion}\label  {sec:conclusion}
This work has defined a lightweight application, Intethi-Perf for collecting network measurements in a community network.
Since the application is lightweight in terms of data usage, this means that the network won't be clogged up with a lot of measurements data.
We saw how to mitigate the legal issues that come with recording personal user data through hashing and not using exact user location.
We also saw in this work how building a system using an agile approach can be helpful in dealing with changes that come during the development process.

\section{Future Work}\label{sec:future-work}
This work just showed the beginning of a network monitoring system, using mobile phones as collectors of network measurements.
More work still needs to be put, and the collection of measurement should in the future be done by dedicated Wi-Fi enabled micro-controllers and not be left for the phones.
Adding Wi-Fi enabled hardware to run measurements and not rely on phones will make realtime data collection possibly.
This will also ensure that we will also always have devices to collect measurements just in case one of the stakeholders want to run or schedule immediate measurements.
We can also implement machine learning algorithms to help automatically monitor the network and take precautions as needed.
These can be put in the analyzer component of the system and will analyze all recorded data so to detect any anomalies within the network.

