\section{Introduction}\label{sec:introduction}
Community networks are large-scale decentralized networks that are built and operated by the citizens\cite{Selimi:2014:TAD:2723218.2723265}.
The networks are at times run by non-profit organizations who can cooperate with stakeholders to develop community services, which include the internet and local networking\cite{Braem:2013:CRC:2500098.2500108}.
These networks are becoming more and more popular for providing Internet access to remote areas.
They are ideal as they are distributed and meaning there is no single point of failure and  as they can be managed by the users of the network.
As these networks are mostly managed by the users in the network it will be good to have a monitoring system that users, and the providers of the network could have used to look at how the network is performing.
As a result this literature review looks into Community networks, Network measurement tools and network monitoring tools that have already been established for other different types of networks.
The end result is to investigate what other research has been done for Quality of Service monitoring systems for other networks and get a good idea into the good practices that one should consider when building a monitoring system.
\paragraph{}
This literature, also looks into how other systems best store network measurements as well as how they perform Anomaly detection using Machine Learning and network measurement data that was collected.
We hope by the end of the review we will have a clear picture of how different network measurement systems are developed, which machine learning algorithms can be used to detect anomalies in the network given some data and the architectures of these systems.
We lastly take a brief look into the different methods that are used to visualize measured data that will be accessible to all users to view the network activities.
For each section we look into different research on that section comment on them and then look into how we can potentially use these ideas into building a QoS monitoring tool for community networks.
