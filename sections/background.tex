\section{Background}\label{sec:background}
%Are there any theories, concepts, terms, and ideas that may be unfamiliar to the target audience and will require you to provide any additional explanation?
%Any historical data that need to be shared in order to provide context on why the current issue emerged?
%Are there any concepts that may have been borrowed from other disciplines that may be unfamiliar to the reader and need an explanation?
Quality of Service(QoS) is defined to be the ability of the network to provide a service at an assured service level\cite{soldani_book}.
QoS has relations to the term Quality of Experience(QoE), which explains how users experience a particular service\cite{6462223}.
For example if we have the network throughput to be low, and we were watching a YouTube video, we will not have a good experience of this service.
In this case the raw throughput measured will be the Quality of Service measurement while the experience that the users have with streaming the YouTube videos will be Quality of Experience.
The above example is illustrates that QoS affects the users QoE of services.
Quality of Service includes metrics like throughput and delays being experienced in the network.
For QoS metrics to be meaningful and effective each data point needs to be be tied to a particular location, device that recorded the data, and possibly the time the data was collected.

The aim of this work is to build a software tool that would allow different stakeholders in the network to monitor these metrics in order to estimate the Quality of Service that can be expected from the network.
%Community Networks section
\subsection{Community Networks}\label{subsec:community-networks}
%From a technical point of view, community networks are
%large-scale, distributed and decentralized systems composed
%of many nodes, links, content and services.
Community networks are defined to be large-scale self-organized distributed and decentralized networks which are built and managed by the citizens for citizens \cite{Braem:2013:CRC:2500098.2500108}.
The architecture of community networks is usually built with low cost hardware with the nodes often running an open source software\cite{Braem:2013:CRC:2500098.2500108}.
Nodes in a community network are devices that relay data within the network or devices that provide other nodes network access\cite{8320771}.
The links between the nodes can be wireless or wired\cite{8320771}.
Community Networks are part of the subset of networks characterized as:
(i)open because everyone has a right to know how they are built,
(ii)free because networks access is not based on discriminatory principles
(iii)neutral as the network can be used to transmit data of any kind and by any participant\cite{Braem:2015:AEQ:2830629.2830639}.

\paragraph{}The setup of community networks make them more appealing for providing internet access to unrepresented areas.
Some characteristics that allow for this are, decentralization of the network\cite{Selimi:2014:TAD:2723218.2723265} which removes the chances of having a single point of failure within the network.
Community networks are also self-managing\cite{Braem:2013:CRC:2500098.2500108}, which means that they will not be so costly as they will not require a body to manage the network since the networks are managed by the users of the network.
Examples of community networks include, \textbf{Inethi Network} whose goal is to extend community network models by enabling sharing and authoring of local digital resources and services\cite{inethi_technologies}.\textbf{Zenzele.net} which is a community network that provides cheap telephone and internet access to the community members of the\cite{zenzeleni.net}.
\textbf{guifi.net} that operates in Spain\cite{guifi}, which by 2015 was the world's largest community networks in terms of number of nodes and coverage with over 27000 operational nodes\cite{2015:TOG:2852375.2852741}.

\subsection{Related Work}\label{subsec:prior-research}
\subsection{Monitoring systems}\label{subsec:monitoring-systems}
Monitoring systems are essential to network systems as they give an overall view and many metrics for a network.
Monitoring of networks is crucial not only for the end users of the network who need to know the performance metrics of the network, but also for the policy-makers and the network regulators\cite{7523537}.
Three stakeholders in network performance measurements are the end-users, the ISPs and the regulators\cite{7076582}.
All these groups of people all have different use cases for the measurement data.
For instance, ISPs can use the data to identify, and isolate network faults and to have an understanding of user's end to end service experience\cite{Ford:2018:RWR:3243157.3243167}.
Data collected from monitoring networks by running certain measurements is not only important to improve user experience on the network, but also important for providing feed back on the design of other upcoming technologies\cite{7523537}.
\paragraph{}
Monitoring of networks consists of being able run passive and active measurements on the networks.
Different measurement types include ping measurement where we measure the time taken to ping a target and if there is any packet loss during the ping process, TCP throughput, this is in two forms, the download speed and the upload speed of the network, DNS lookup, which tells the average time it takes to be able to resolve a name of a host to its ip address, lastly we have http measurements telling us the time it takes to send and get a http response to a known http server.
Different systems have been built thus far to take networks measurements.
Some of these are accessible to end-users like by going to their website, like OOKLA Speedtest\cite{7523537}.
Tools like OOKLA Speedtest have been criticized for not being scalable and not repeatable\cite{7523537}.
Repeatability has to do with the fact that we cannot continuously take measurements on the network.
However, there have been more internet measurement platforms that have been introduced through research.
Internet measurements platforms refers to infrastructure that has been dedicated to periodically run network measurement test on the internet\cite{7076582}.
These platforms are built on wifi-enabled micro controllers, which are always actively taking measurements.
\subsubsection{Measurement Tools}\label{null:measurement-tools}
There have been different network measurement tools designed for collecting network measurements for community networks in particular.
Some of these studies where mentioned and discussed by Braem and team\cite{Braem:2015:AEQ:2830629.2830639} and include:
\begin{itemize}
    \item RIPE ATLAS, which is a measurement infrastructure consisting of thousands of hardware probes and servers all around the globe.\cite{7076582, Bajpai:2015:LLU:2805789.2805796}.
    While hardware probes obtain active measurements to determine network connectivity and global reachability, anchors serve as dedicated servers that can act as sources and sinks for the network measurement traffic\cite{Bajpai:2015:LLU:2805789.2805796}
    \item Another tool is the BISMark tool that seeks to measure home broadband performance by means of custom gateway firmware\cite{Braem:2015:AEQ:2830629.2830639}
    \item Perfsonar-ps, a suit of measurement tools, which can be deployed freely\cite{Braem:2015:AEQ:2830629.2830639}.
\end{itemize}
\subsubsection{Network Measurement Platforms}\label{null:network-measurement-platforms}
\subsubsection{Using Smart phones for network data collections}
One approach to taking network measurements is to make use of the users mobile phones constantly take network measurements and then sending these to a central database server for analysis.
This can be via an application installed on mobile phones that will collect measurements on fixed time intervals.
This approach has problems, which include privacy issues, power demands of measurements that will affect the mobile phone's battery life to mention but a few.
LiveLab, is a solution to address some challenges that arise from using mobile phones in taking network measurements\cite{Shepard:2011:LMW:1925019.1925023}.
Livelab is method for measuring real-world smart phone usage and wireless networks with a re-programmable in device logger, which is built to overcome privacy and power challenges\cite{Wang:2015:MMA:2757290.2757291}.
Livelab leverages hashing of the data recorded from the phone to ensure overcome the privacy issues\cite{Shepard:2011:LMW:1925019.1925023}.
For power saving, Livelab uses 4 different logging methods, which are\cite{Shepard:2011:LMW:1925019.1925023}, Interrupt depended on logging as opposed to polling method.
Another method is, Piggy-backing, which is to save all the data on the phone and then collecting them when the phone is connected to power and idle
Optimizing logging intervals for periodically logged items is also a method used.
Lastly having the logger hitchhike on other system application and services waking up.

\subsubsection{MONROE PLATFORM}
MONROE platform is a multi-home open access hardware-based platform for running measurements and experiments in operational mobile broadband networks\cite{7523537}.
As of date MONROE platform is said to consist of 250 nodes, both mobile and stationary\cite{8002921}.
Each node runs 3 software, the management software to ensure that the node  always remains operational and to enable remote updates to all the other software components.
Another component is the maintenance software that monitors the operational status, and the last software is an experiment enabler\cite{8002921}.
\paragraph{}
Monroe can handle experiments that range from continuous latency measurements to real-time flows together with meta-data like location of data type of device for measuring data and so on\cite{7523537}.
The platform comes with a website for viewing the measurements being collected in the network, Nodes and software for orchestrating the collection of the measurements\cite{7523537}.
\subsubsection{Measurement-Lab and Community Lab}
Measurement lab(M-Lab) is "an open, distributed server platform for researchers to deploy active internet measurements tools" \cite{Dovrolis:2010:MLO:1823844.1823853,Braem:2015:AEQ:2830629.2830639} with the data collected released to the public domain\cite{Dovrolis:2010:MLO:1823844.1823853,Braem:2015:AEQ:2830629.2830639}.
M-Lab only runs active measurement tests and, in addition conducts measurements between the user/client, and the M-Lab servers to examine the end-end performance along the path\cite{Dovrolis:2010:MLO:1823844.1823853}.
Some tools provided by M-Labs currently include:
\begin{itemize}
    \item \textbf{Network Diagnostic Tool (NDT)} which is a tool that measures the through put between the client and host in terms of download and upload speeds\cite{Dovrolis:2010:MLO:1823844.1823853,Braem:2015:AEQ:2830629.2830639}.
    The tools also tries to determine the causes of slow speeds as well as checks for proxies, NAT devices or middle boxes between the machine running the tests and the M-Lab server collecting the tests thereby providing several objective indicator of user experience of an internet connection\cite{Braem:2015:AEQ:2830629.2830639}.
    \item \textbf{Network Path and Application Diagnosis (NPAD)} which uses TCP to measure end-to-end throughput and information about the switch/route queues along the path\cite{Dovrolis:2010:MLO:1823844.1823853}.
\end{itemize}
\paragraph{}
Community-lab is said to be an open infrastructure that provides to researchers and experimenters a testbed to carry out experiments within wireless community networks\cite{Rameshan:2013:MSC:2508222.2512838}.
It comes with a testbed that support experimentally-driven research on community based networks\cite{Braem:2013:CRC:2500098.2500108}.
The testbed provided by Community-Lab was inspired by Planet lab\cite{7076582,Braem:2015:AEQ:2830629.2830639}.
Each node in Community-Lab consists of two to three devices, and these include the community device, the research device and an optional recovery device connected together via a local network\cite{Rameshan:2013:MSC:2508222.2512838}.
Community devices are wireless routers, research devices are low powered systems running OpenWRT distribution that makes it possible to allow simultaneous virtual containers\cite{Rameshan:2013:MSC:2508222.2512838}.