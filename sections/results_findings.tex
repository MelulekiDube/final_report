\section{Results and Findings}\label{sec:results-and-findings}
This sections seeks to speak into the evaluation the application.
One of the non-functional requirements was to design a lightweight application in terms of number of resources needed by the application.
The resources that were focused on include storage space taken on the phone, battery used during operation, and the data used in terms of internet.
To figure these out the application was left running o the phones over three days.
After the three days we then checked to see the data that the application was used, the battery used, and the amount of extra storage that was used.

\paragraph{}
The application on its own without any extra data takes up 6.7MBs of storage.
Initial battery level before running the experiments was 100%.
Initial data used by the application was at 0 mbs.

The settings on the application are as follows:
\begin{itemize}
    \item Checking in time was 2 minutes.
    This means that every two minutes the phone will check up with the server to get new measurement jobs to run.
    The phone will thus have 3 new measurement jobs to run every two minutes.
    \item The maximum data to be used capacity was set to 50 MBs.
    That means that the application could use only upto 50MBs after which it will be paused.
    \item Application sent user personalized data every 2 minutes.
\end{itemize}
The reasons for the above conditions is because stress testing was what was being aimed for.
The server had 6 jobs and every time the phone checked in it got these 6 jobs.
The results are presented below.

\subsection{Results}\label{subsec:results}
\begin{table}[h!]
    \begin{center}
        \caption{Data showing how much results were used by the phone.}
        \label{tab:results}
        \begin{tabular}{l|c} % <-- Alignments: 1st column left, 2nd middle and 3rd right, with vertical lines in between
            \textbf{Resource} & \textbf{Result}\\
            \hline
            Storage Taken & 6.8MBs \\
            \hline
            Data used & 6 MBs \\
            \hline
            Battery Capacity & 80\% \\
            \hline
        \end{tabular}
    \end{center}
\end{table}
Table \ref{tab:results} showing the results from running the application for the 3 days with the above mentioned settings.
As can be seen, the application is not really resource intensive.
We managed to use 6 MBs in data usage.
This 6 MBs used includes the data used for running measurement tests, sending personalized user data also checking for extra data.
Therefore, we can see from that in terms of internet usage the application is lightweight as was one of the non-functional requirements.
The application is also efficient in terms of storage resource needed.
This conclusion is reached as the application required extra 1 MBs for storage.
The 1 MBs used was for probably for storing the user preferences.
Aside the preferences the application does not store any more details.
Lastly the application also uses the battery efficiently.
After being left to run for 3 days as the only main application on the phone, the application used 20\% battery.
The daily averages for the resource used are estimated to be as follows, for storage usage  0.3MBs, for data Usage 2MBBs and for battery used 6\%.
Therefore, it is fair to conclude that the non-functional requirement relating to the application being lightweight was met as not much resources are being utilized by the application.