\section{Requirement Analysis and Design}\label{sec:design-decision}
\subsection{Requirements Analysis}\label{subsec:requirements-analysis}
\begin{itemize}
    \item Lightweight application to run on users phones to run data collection system
    \item Testbed for researchers to be able to run experiments on the network
    \item A way of showing users how many data is used on inethi
    \item Be able to analyze ip addresses sending the most request, and most visited ip addresses
    \item Users to be able to track data usage from their children's accounts
    \item Give users ability to view data accross differnt times
\end{itemize}
\subsection{Design}\label{subsec:design}
\begin{figure*}
    \begin{center}
        \includegraphics[width=1\linewidth]{resources/system_des.png}
    \end{center}
    \caption{Showing the system overview, different components making the system and the communication directions between these components in the systems.}
    \label{figure:system}
\end{figure*}
Figure\ref{figure:system} shows the architecture of the system with the different components that make up the system.
The main components of the system as from the design are:
\begin{itemize}
    \item \textbf{Collector}, which is the has the aggregating measurement collected from different collection points and sending these to analyzer for some analysis before we save these to the database.
    \item \textbf{Orchestrator}, used by researchers, system administrator and the system itself to schedule measurements that will be run on the data collection points.
    \item \textbf{Web Interface}, tool used stakeholders to interact with the system as a whole by doing tasks as, scheduling measurements viewing specific metrics in the network.
    \item We also have 3 databases that are going to be used with the system.
    The measurement database and the metadata databases are adopted from the works done by\cite{7523537} with MONROE platform.
    We added the user database to be able to store user accounts so users can be able to also have their network usage data stored for later viewing.
\end{itemize}
\paragraph{}
Coming up with the system required number of decision to be made.
One of these was the database types to use for the 3 databases that we want to have in the system.
The three databases we have in the system have different use cases and this allowed us to have different databases for each of those.
\begin{itemize}
    \item The user databases needed to hold user account in a way that it was going to be easy to query user details based on a unique user id or username.
    We also are going to need a database type that is simple to design while allowing fast queries of a specific username, we decided that the database type to use for this database is going to be a document store database.
    A document store database is a database type where all the entries are stored in documents with the documents addressed using unique keys\cite{nosql_dbs}.
    These databases offer great performance and horizontally scalability options\cite{nosql_dbs}.
    In this case they are attractive for their great performance as we need to be doing fast queries.
    We settled for Mongo Database as it has a lot of support and a very well documented Java library.
    \item The Metadata Database is a database that we are going is used to store the experiment request by researchers on the network.
    Therefore, when a researcher requests particular experiments to be done on the system we want to be able to store this request.
    The experiment request stored will rarely be queried but when we do so, we still  need the queries to be as fast as possible.
    Experiment results are all be given a random unique key, which we will use to run queries.
    Hence, for the above reasons we can still use a document store database as it fits well with our requirements.
    For this we also chose the Mongo Database for consistency and for the reasons already mentioned.
    \item Lastly we have the measurement database, which is a database that we will be storing all recorded measurements.
    This database stores data indexed by timestamps as we do not need any primary keys to reference measurement.
    Therefore, to satisfy this requirement a time series database is going t be more suitable.
    We then decided to go with Influx database as its one of the most well-supported time series database with a good documentation.
    Influx also allows us to tag some other parameters in every measurement that we store so that when we query we can have fast queries for these parameters.
    An example is that sometimes users will want to see measurements from a particular location.
    To ensure that this operation is going to be fast and done in Constant Time Big O notation, we will then make location parameter a tag.
    This alone will change the performance of searching for measurements in particular locations.
\end{itemize}
%still need to add more design decisions
