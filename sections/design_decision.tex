\section{Requirement Analysis and Design}\label{sec:design-decision}
This paper focuses mostly on designing the tool for collecting measurements from the phones and storage from these measurements, there will be limited talk about users hence the model diagrams are more concerned with the system and not user interaction with the system and there is limited interaction with the users from collection of data and storage of data point of view.
\subsection{Requirements Gathering}\label{subsec:requirements-gathering}
To ensure that a software system is useful to all those involved there is need to gather requirements from the different types of users of the system.
In our study we gathered these requirements through few techniques.
\subsubsection{Communication with Project Supervisor}
The initial requirements were given by the project supervisor.
Throughout the project we consistently checked in with the supervisor via emails and meetings to ensure that the system being built was per requirements and to gather any ad hoc requirements that might have come up.
\subsubsection{Communication with the Inethi Network Stakeholders}
The main stakeholders for the software system are, Network administrators of Inethi network, potential researchers in the community and the users of the Inethi network.
The stakeholders we interacted with included the directors of the community network, the administrators of the network and the normal users of the network.
For directors of inethi we had face to face encounter with them were we briefed them about the project, and we asked their opinions on it.
This meeting enabled us to establish the requirements from the point of view of the directors of the network.
Their requirements where adding functionality to be able to see the mostly requested pages.
Another request was being able to see the top ip addresses in terms of the ip address downloading and uploading the most.
\paragraph{}
We also met with the potential users of the network and administrators of the network.
This was done through a workshop organized for users to interact with some mock-ups we had designed.
For the workshop, we interacted with 3 users of the Inethi network and 1 person having an admin role within the network.
We also interviewed the participants about the idea of the project and asked for their thoughts and if there were any features that they needed from the application.
The requirements that came from the potential users and network administrators included adding functionality for users to be able to monitor how their apps use the network, allowing parents to monitor the website visited by their children also allowing the administrators to filter the results shown in the graph based on different parameters.
\subsubsection{Surveys}
We also asked users of the network to answer some survey question at the end of the workshop.
This also gave us more feedback that the users might have not mentioned during the sessions.
The answers to these surveys were anonymous.
\subsection{Requirements Analysis}\label{subsec:requirements-analysis}
Using the above techniques we managed to gather a list of requirements and have split these into two sets, the one set is a list of functional requirements, and the other set of is that of non-functional requirements.
The list of functional requirements includes:
\begin{itemize}
   \item Be able to analyze IP addresses sending the most request, and most visited ip addresses
    \item Users to be able to track app data usage from their phones.
   \item Users to be able to track data usage from their children's accounts
    \item A way of showing users how much data is used on inethi
\end{itemize}
The list of non-functional requirements includes:
\begin{itemize}
    \item Lightweight(in terms of used resources) application to run on users phones to run data collection.
    \item Secure ways of storing user data to avoid leaks of collected data.
    Secure speak to two things, data should not be accessible from outside the system and if it happens, the data should be stored such that the identity of users is not compromised.
\end{itemize}
\subsection{Design}\label{subsec:design}
\subsection{Package Diagrams}\label{subsec:package-diagrams}
From the high level system design and the requirements we got form the users of the system we were able come up with models of the system, which later on where used to drive development of the system.
One is a package diagram and shows the arrangement and organizations of model elements.
\begin{figure*}
    \begin{center}
        \includegraphics[width=1\linewidth]{resources/package_diag.png}
    \end{center}
    \caption{Showing UML package diagrams of the system.}
    \label{figure:packages}
\end{figure*}
Figure \ref{figure:packages} shows the package diagram of our system.
The server in our system contains 5 packages that all represent separate services.
The server package can be hosted within the Inethi network server and will then be accessible by the other packages like the web interface and the mobile devices.
All the other packages use one, or many of the services offered by the server.
An example is the mobile phones, which use the Database Request Handler service to save the recorded data to the database.
The reasons for prohibiting direct access to databases from mobile phones is so that we can analyze the data being written to the database to ensure that all data is safe.
Also, this allows for us to change the underlying databases without needing to change or update any functionality on the mobile phones.

\subsection{State and State Transitioning of the System}\label{subsec:state-and-state-transitioning-of-the-system}
This section details the different states that the mobile application can be in and how the application will alternate between these statuses.
\begin{figure*}
    \begin{center}
        \includegraphics[width=1\linewidth]{resources/state_chart.png}
    \end{center}
    \caption{Showing State Machine Chart for the mobile application that we will be building}
    \label{figure:state}
\end{figure*}
Figure \ref{figure:state}  shows the state chart of the mobile application that is to be built.
The phone has 3 main states shown in the Live container, and these are 'idle', 'checking_in' and 'running measurements'.
The 'idle' state is the first state the phone enters upon launching the application.
In this state the phone keeps a timer, which triggers the checking in, and if necessary running of measurements.
During checking in, the mobile application contacts the server to find out if there are any experiment jobs that the phone can run.
If the server responds with a job, the phone will then schedule these measurements and return to idle state while starting the timer.

If time comes for a particular measurement to run the phone will leave the idle state and move to 'run measurement' state.
During the 'run measurement' the scheduled measurements are run, and the results are built and sent immediately to the server as required.
The phone then starts the timer and goes back to being idle.
Every time the phone leaves the idle state the timer will be stopped and resumed when it comes back.
The phone will forever be in this loop unless the user terminates the application directly by which the application will then stop.
\subsection{System Architecture}\label{subsec:system-architecture}
\begin{figure*}
    \begin{center}
        \includegraphics[width=1\linewidth]{resources/system_des.png}
    \end{center}
    \caption{Showing the system overview, different components making the system and the communication directions between these components in the systems.}
    \label{figure:system}
\end{figure*}
Figure\ref{figure:system} shows the architecture of the system after analysing the given requirements with the different components that make up the system.
The main components of the system as from the design are:
\begin{itemize}
     \item \textbf{Orchestrator:} used by researchers, system administrator and the system itself to schedule measurements to be run on the data collection points.
     Admins and researchers will access the collector from a web interface.
    \item \textbf{Collector:} aggregates measurements collected from different collection points and sends these to the server for some analysis and storage.
    \item \textbf{Web Interface:} tool used stakeholders to interact with the system as a whole by doing tasks as, scheduling measurements viewing specific metrics in the network.
     The scheduled measurement are submitted to the orchestrator.
    \item \textbf{Data storage:} which consists of 3 databases that are going to be used with the system.
    The measurements database and the metadata databases are adopted from the architecture of the with MONROE platform\cite{7523537}.
    We added the user database to be able to store user accounts so that users can be able to have their network usage data stored for later viewing.
\end{itemize}
\subsubsection{Storing of Network Measurement Data}
Storage is one important issue that comes up when researchers collect data and try to analyze the data.
The storage of data affects how easy it is going to be to extract the data for analysis and to what extent one can be able to run machine learning algorithms on the data.
In this project, data will be collected from different devices and different applications(phone application and computer application).
As such, there is need to have a good design for the data storage.
\paragraph{}
One idea to consider is taken from MONROE\cite{7523537}, where the system consisted of a Remote repository and Data Importer, which will remotely collect experimental data from each node after every experiment.
It also consists of Database System, which consists of two tables one for experiment measurements, and the other for meta-data like location device type.
The database that is used in MONROE is the Cassandra database.
Casas and team also used the Cassandra database in their paper for storage of data for the reasons that Cassandra is a fully distributed database with no single point of failure\cite{8255998}.
\subsubsection{Databases}
Coming up with the system required number of decision to be made.
The three databases we have in the system have different use cases and this allowed us to have different databases for each of those.
\begin{itemize}
    \item The user databases needed to hold user Profile information in a way that it was going to be easy to query user details based on a unique user id or username.
    We needed a database type that is simple to design while allowing fast queries of a specific user profile.
    We therefore, decided that the database type to use for this database is going to be a document store database.
    A document store is a database type where all the entries are stored in documents with the documents addressed using unique keys\cite{nosql_dbs}.
    These databases offer great performance and horizontally scalability options\cite{nosql_dbs}.
    In this case they are attractive for their great performance as we needed to be doing fast queries.
    We settled for Mongo Database as it has a lot of support and a very well documented Java library.
    \item The Metadata Database is a database that is used to store experiment requests by researchers on the network.
    When a researcher requests particular experiments to be done on the system, we needed to be able to store this request.
    The experiment request stored will rarely be queried but when done the queries need to be as fast as possible.
    Experiment results are given a unique key, which we will use to run queries.
    For the above reasons we decided to use a document store database as it fits well with the requirements.
    For this, we also chose the Mongo Database for consistency and for the reasons already mentioned.
    \item Lastly we have the results' database, which is a database that we will be storing all recorded measurements results.
    This database stores data indexed by timestamps as we do not need any primary keys to reference measurement.
    Therefore, to satisfy this requirement a time series database is going t be more suitable.
    We then decided to go with Influx database as its one of the most well-supported time series database with a good documentation.
    Influx also allows us to tag some other parameters in every measurement result that we stored so that when queried there can be fast queries for these parameters.
    An example is that, sometimes, users will want to see measurement results from a particular location.
    To ensure that this operation is going to be fast and done in Constant Time Big O notation, that is number of comparisons done is not depended on the number of items in the database, we made location parameter a tag.
    This alone will change the performance of searching for measurements in particular locations.
    Another tag is the job ID tags for results that corresponded to measurements jobs submitted by researchers
\end{itemize}
Since all the databases used in this case are noSQL databases we did not have to design the schemas of the databases.
\begin{table}[h!]
        \caption{Table showing a table view of the Users database, highlighting the properties stored in the database.}
        \label{tab:usertable}
        \resizebox{\columnwidth}{!}{%
        \begin{tabular}{c|c|c}
            \hline
            \textbf{Email Address} & \textbf{Hash of Email Address} & \textbf{User Type}\\
            \hline
            String & String & String\\
            \hline
        \end{tabular}
        }
\end{table}
Table\ref{tab:usertable} table view of the users' database.
It highlights the properties from users that are going to be stored, which include their email address, the MD5 hash of the email address.
The rationale for this design is because we do not need to be authenticating the users ourselves.
Since users are going to use their email addresses, we can use the authentication services by google and others to do the authentication for us.
This will provide more security for users as we don't get to store their passwords, and as seen above, simplifies the design of the system.
As mentioned the system will be recording usage data, and to maintain anonymity, we will be storing user data with the hash value of the email address and not their email address to ensure that the data is anonymous and cannot be traced back to the users.
For retrieving user data, the users will provide their email address, and in turn the system will check within the database to get their hash value and then use this hash value to search for any data that belongs to them.
\begin{lstlisting}[language=json, caption=JSON structure showing how the experiment metada documemts stored in the database will look like]
{
"request_type":string
"job_description":{
"measurement_description":{
"type":string
"key":string
"start_time":string
"end_time":string
"interval_sec":int
"count":long
"priority":long
"parameters":{
//depends on the measurement
}
}
"node_count":int
"job_interval":int
}
"user_id":string
}
\end{lstlisting}
Listing 1 shows the structure of an individual document that will be stored in the experiment metadata database.
The parameter fields will have different structure for different measurement types that are being scheduled.
The different possible fields are shown for each measurement type in the appendix.
Each job will have an ID field, which is randomly generated and unique for all the jobs scheduled.
The user id filed is needed to know, which jobs belong to which users.
Here, we will also hash this value with the email address of the researcher who scheduled the job.
%still need to add more design decisions
\begin{figure*}
    \begin{center}
        \includegraphics[width=1\linewidth]{resources/Measurements.png}
    \end{center}
    \caption{Showing UML class diagrams of the different types of measurements and their associations.}
    \label{figure:measurement}
\end{figure*}

Figure \ref{figure:measurement} shows us the UML Diagrams for the different measurements that can be stored and how they are associated to each other.
As can be seen all measurements types inherit from the Measurement class, which has the properties task_key, time, username, and whether a measurement is an experiment measurement result or not.
Since we are using Influx Database for storing our data, all measurement data stored needs to be referenced by the time as Influx is a time series database.
We are storing the username so that we can know from the database how the number of different users/nodes we are getting the results from.
The username is hashed at all times to also maintain anonymity.
The property for whether a measurement is related to an experiment stems form the fact that we also record some manual measurements that mobile phone users are going to trigger from the phones.
These can also be the realtime measurements that we are constantly collecting to keep track of the quality of the networks.
The experiment measurements are those that belong to a particular researcher and thus are as a result of a researcher scheduling measurements on the system.
The task_key of the measurements is because we want to be able to monitor all jobs that are run and be able to tie them back to the people who scheduled the measurements so when they query them we can give them the measurements back.
The task key is a unique for all the measurements.