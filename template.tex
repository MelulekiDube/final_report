\documentclass[plain]{sigplanconf}
\usepackage{balance} % For balanced columns on the last page
\usepackage{amsmath}
\usepackage[T1]{fontenc}
\usepackage{lmodern}
\usepackage{graphicx}
\usepackage{amssymb}
\usepackage{tikz}
\usepackage{array}
\usepackage{longtable}
\usepackage[
bookmarksopen,
bookmarksdepth=2,
breaklinks=true
]{hyperref}
\makeatletter
\def\BState{\State\hskip-\ALG@thistlm}
\makeatother

\makeatletter
\def\@copyrightspace{\relax}
\makeatother
\begin{document}
	\title{Language identification for South African Bantu languages Using Rank Order Statistics}
	%%\subtitle{Philosophy Invades Software Engineering}
	\authorinfo{Meluleki Dube}
	{Department of Computer Science\linebreak University of Cape Town\linebreak South Africa}
	{June 2018}
	\maketitle
	\begin{abstract}
		A lot of research has been done in the field of language identification. However, only a small proportion of these methods used for classification have been tested with the Bantu languages. In this paper we then look at one of the methods, and how it performs for the Bantu languages. The method used in this research is n-gram counting using rank orders. Using this method, we investigated how varying the testing chunk size and learning size affected the accuracy of correctly identifying the languages. The highest accuracy obtained was 99.3\% with testing size of 495 characters and learning size of 600000 characters. The lowest accuracy obtained was 78.72\% when the testing size was 15 characters and learning size was 200000 characters.  
	\end{abstract}
	\begin{CCSXML}
		<ccs2012>
		<concept>
		<concept_id>10002951.10003317.10003338</concept_id>		
		<concept_desc>Information systems~Retrieval models and ranking</concept_desc>
		<concept_significance>500</concept_significance>
		</concept>
		</ccs2012>
	\end{CCSXML}
	\keywords
	N-grams, N-fold cross validation, Rank Order statistics
	
	\section{Introduction}\label{sec:introduction}
Community networks are large-scale decentralized networks that are built and operated by the citizens\cite{Selimi:2014:TAD:2723218.2723265}.
The networks are at times run by non-profit organizations who can cooperate with stakeholders to develop community services, which include the internet and local networking\cite{Braem:2013:CRC:2500098.2500108}.
These networks are becoming more and more popular for providing Internet access to remote areas.
As these networks are mostly managed by the users in the network it is good to have a monitoring system that users, and the administrators of the network could have used to look at how the network is performing.
The performance of the network will be based on network metrics.
The networks metrics to be used are time for HTTP response to come back after sending an HTTP request, a DNS lookup speed test for finding our how long it took for a DNS lookup to finish, a TCP throughput speed test, which measures the Download speed and Upload Speed, ping test, which measures the time taken for pinging a chosen target and lastly trace-route values give the different servers that the system hits when sending packets and time taken to reach any of them.

This paper then focuses on building a Monitoring tool for the community networks with Inethi as a case study.
There is more emphasis on creating the application for actually collecting the data from the network, and the storage of the data that is collected.
The paper also describes an agile process of building a system and different techniques for testing an application.
The rest of the paper describes the requirements from the stake holders' perspective, then describes the design and implementation of the system, lastly we describe the testing and results of the application performance.
	\nocite{*}
	\bibliographystyle{abbrvnat}
	\bibliography{references}
\end{document}